\section{Question 5}
Consider the dataset of 3000 bivariate, labeled samples in \texttt{data.csv} file. 
Firstly, split the dataset into training (70\%) and test set (30\%).

\subsection{Item a}
Assume the samples labeled with ``$+1$'' are drawn by a bivariate gaussian density 
with parameters $\mu_{+1}$, $\Sigma_{+1}$, while the samples labeled with ``$-1$'' 
are drawn by another gaussian density with parameters $\mu_{-1}$, $\Sigma_{-1}$. 
The prior probabilities $P(C_{+1})$ and $P(C_{-1})$ are unknown as well. 
Estimate the missing parameters (with the training set), 
present the discriminant functions of the Bayes classifier (MAP criterion) 
and evaluate its performance (accuracy, precision, recall) over the test set.
Comment all your results.

\subsection{Item b}
Calculate the decision boundary and plot it on a graph with the samples of the test set.
Comment your results.

\noindent\rule{\textwidth}{.5pt}

The dataset in \texttt{data.csv} consists of three columns, 
two for the features and one for the label. 
In the following, the first feature's value is called $x_1$, and the second, $x_2$.
%
Since the classifier will be selecting one of two classes for each sample given to it,
it is a dichotomizer, and in particular performance will be measured with respect to class $+1$.

A first step in creating the dichotomizer is readying the data. 
Pre-processing actions like shuffling samples and normalizing features 
are common before making the train-test (or train-validation-test) split. 
%
Since the features have relatively small ranges and the Bayesian classifier uses data as-is,
normalizing is not necessary.
%
As for if shuffling is desirable can be easily determined visually from the data distribution,
since the number of features is small. 
\Cref{fig:shuffling} shows how shuffling affects the original data and the 70\%-30\% train-test split.
%
\begin{figure}[hb]
    \centering
    \caption{Effects of shuffling and resulting data split. Seed: 242104677.}
    \begin{subfigure}{.49\textwidth}
        \centering
        \caption{Shuffling effects on data and split.}
        \label{fig:shuffling}
        \includegraphics[width=\linewidth]{../../python_code/results/bivariate_classifier/shuffling_effect.png}
    \end{subfigure}
    %
    \begin{subfigure}{.49\textwidth}
        \centering
        \caption{Train-test split.}
        \label{fig:split}
        \includegraphics[width=\linewidth]{../../python_code/results/bivariate_classifier/split.png}
    \end{subfigure}
\end{figure}

The clustering of dark blue points in the original data indicates that 
label $+1$ tended to happen more frequently towards the end of the dataset.
As such, splitting the data without shuffling would likely skew the prior of the training portion,
biasing the classification process%
\footnote{%
    The discussion on the classifier's results present at the end of this answer shows that 
    using the original data leads to one more misclassification than using the shuffled data.
}.
Therefore, shuffling the samples before the split is adequate.
\Cref{fig:split} shows the split applied on the shuffled samples to obtain the train and test sets.

For the dichotomizer to work, classes priors and likelihoods must be known.
Since they aren't, estimation is required.
%
Given that the likelihoods are gaussian by hypothesis, 
this means estimating each class' mean and covariance.
%
Let
%
\begin{multicols}{2}
    \begin{itemize}
        \item $P(C_{+1})$ be the prior of $C_{+1}$;
        \item $\bm\mu_{+1}$ be the average of $C_{+1}$;
        \item $\bm\Sigma_{+1}$ be the covariance of $C_{+1}$;
        \item $P(C_{-1})$ be the prior of $C_{-1}$;
        \item $\bm\mu_{-1}$ be the average of $C_{-1}$;
        \item $\bm\Sigma_{-1}$ be the covariance of $C_{-1}$;
    \end{itemize}
\end{multicols}
%
then, using the convention that a hat indicates an estimator,
%
\begin{align}
    \widehat{P}(C_i) &= 
        \frac{N_i}
             {N_{+1} + N_{-1}} 
\label{eq:prior}\\
    \widehat{\bm{\mu}}_i &= 
        \frac{1}{N_i}
        \sum_{n=1}^{N_i} \mathbf{x}^n_i \\  
    \widehat{\bm\Sigma}_i &= 
        \frac{1}{N_i - 1} 
        \sum_{n=1}^{N_i} 
            (\mathbf{x}^n_i - \widehat{\bm\mu}_i)
            (\mathbf{x}^n_i - \widehat{\bm\mu}_i)^\top 
\label{eq:cov}
\end{align}
%
where 
$i$ indexes a class,
$N_i$ is the number of instances of class $i$ in the training set, and
$\mathbf{x}^n_i \in \mathbb{R}^{2\times 1}$ is the $n$-th sample of class $i$ in the training set.
%
Applying eqs. \eqref{eq:prior}-\eqref{eq:cov} on the training data, 
the following values were obtained.
\Cref{fig:estimators} shows what the estimators look like on top of the training data 
(ellipses encompass three standard deviations).
%
\begin{figure}[hbp]
    \centering
    \caption{Average and covariance estimators fit on training data.}
    \label{fig:estimators}
    \includegraphics[width=7cm]{../../python_code/results/bivariate_classifier/parameters_estimates.png}
\end{figure}

With the prior and likelihoods at hand, defining a discriminant is straightforward.
Of the common formulations, the is chosen the log-ratio of the likelihood-prior products,
since there are only two classes and they're both gaussian:
%
\begin{align}
    g(\mathbf{x}) 
    &= 
    \log\frac{P(C_{+1} | \mathbf{x})}
            {P(C_{-1} | \mathbf{x})} \\
    &= 
    \log\frac{p(\mathbf{x} | C_{+1})}
            {p(\mathbf{x} | C_{-1})} \\
    &=
    -\frac{1}{2}
        (|\widehat{\bm\Sigma}_{+1}| - |\widehat{\bm\Sigma}_{-1}|)
    -\frac{1}{2}
        (\mathbf{x} - \widehat{\bm\mu}_{+1})^\top \widehat{\bm\Sigma}^{-1}_{+1}
        (\mathbf{x} - \widehat{\bm\mu}_{+1}) \nonumber\\
    &\qquad\qquad\qquad\qquad\quad\,\,\,\,\,
        +\frac{1}{2}
        (\mathbf{x} - \widehat{\bm\mu}_{-1})^\top \widehat{\bm\Sigma}^{-1}_{-1}
        (\mathbf{x} - \widehat{\bm\mu}_{-1})
    + \log
        \frac{\widehat{P}(C_{+1})}
                {\widehat{P}(C_{-1})}
\end{align}

Class $+1$ is assigned if $g(\mathbf{x})>0$, and $-1$ otherwise.
The dichotomizer thus operates as indicated in the flowchart of \cref{fig:flowchart};
there, $\mathbf{x}$ is assumed to be a sample from the test set.
%
\begin{figure}[hbp]
    \centering
    \caption{Flowchart of the multivariate MAP classifier.}
    \label{fig:flowchart}
    \begin{tikzpicture}[> = latex]
    % Blocks
    \node [startstop] (start) at (0, 0) {Start};
    \node [io, right=1 of start] (1) {Receive $\mathbf{x}$};
    \node [decision, right=1 of 1] (2) {$g(\mathbf{x}) > 0$ ?};
    \node [process, right=1 of 2] (3) {Assign $C_{+1}$};
    \node [process, below=1 of 2] (4) {Assign $C_{-1}$};
    \node [startstop, right=1 of 3] (stop) {Stop};

    % Arrows
    \draw [->] (start) -- (1);
    \draw [->] (1) -- (2);
    \draw [->] (2) -- (3);
    \draw [->] (2) -- (4);
    \draw [->] (3) -- (stop);
    \draw [->] (4) -| (stop);
\end{tikzpicture}
\end{figure}

\Cref{fig:results} shows the results obtained in testing. 
The red rectangle highlights the area in feature space where all misclassifications occurred.
These are all false positives (i.e., $C_{+1}$ was assigned to sample of class $C_{-1}$).
Remarkably, there were no false negatives, i.e., all $C_{-1}$ assignments were true.
%
\begin{figure}[hbp]
    \centering
    \caption{Classifications using test data from shuffled dataset.}
    \label{fig:results}
    \includegraphics[width=\textwidth]{../../python_code/results/bivariate_classifier/test_errors.png}
\end{figure}

\Cref{tab:results} exhibits three performance metrics for the classifier,
precision, accuracy, and recall, obtained using two different datasets:
the original one, and the shuffled one.
%
As expected from the discussion at the beginning of this answer, 
shuffling improves results (in this case, the precision and the accuracy), albeit only slightly.

\begin{table}[htb]
    \centering
    \begin{threeparttable}
        \centering
        \caption{
            Precision, accuracy and recall of the classifier 
            with and without shuffling samples. 
        }
        \label{tab:results}
        \begin{tabular}{c ccc}
        \toprule
            \multirow{2}{*}{Dataset} & Precision & Accuracy & Recall \\
        \cmidrule{2-4}
            & 
            $\frac{\text{TP} + \text{TN}}
                {\text{TP} + \text{TN} + \text{FP} + \text{FN}}$ &
            $\frac{\text{TP}}
                {\text{TP} + \text{FP}}$ &
            $\frac{\text{TP}}
                {\text{TP} + \text{FN}}$ \\
        \midrule
            Original & 99.22\% & 98.89\% & 100.00\% \\
            Shuffled & 99.33\% & 99.05\% & 100.00\% \\
        \bottomrule 
        \end{tabular}
        %
        \begin{tablenotes}
            \item TP: True Positives, TN: True Negatives,
            \item FP: False Positives, FN: False Negatives
        \end{tablenotes}
    \end{threeparttable}
\end{table}

Looking at the roots of $g$ in \cref{fig:boundary},
the reason behind the misclassifications becomes clear:
some points from class $-1$ had high enough deviation 
in the direction to cross the classification boundary.
%
\begin{figure}[h]
    \centering
    \caption{Boundary of the MAP classifier.}
    \label{fig:boundary}
    \includegraphics[width=\textwidth]{../../python_code/results/bivariate_classifier/boundary.png}
\end{figure}

Since these points fall well within class' $+1$ cluster, 
it would be difficult for any classifier to correctly handle them,
particularly classifiers with convex, smooth boundaries.
%
However, given the shapes and orientations of the train distributions, the boundary observed above seems reasonable.
Moreover, the classifier showed good performance results, proving itself useful during the test.
While whether the classifier would still present good results with further data is unknown, 
it is fair to say that, should bad results appear,
one should consider flaws in the sampling process 
(noise, distributions different than the ones obsreved before, etc)
as much as flaws on the classifier design
(e.g., change criterion, change parameter estimators, chage hypothesis on the shape of likelihoods).